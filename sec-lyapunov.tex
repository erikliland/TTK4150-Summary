%!TEX root = TTK4150-Summary.tex
\section{Lyapunov stability}

\paragraph{Lyapunov function (LF)}
$V(x)$ is an LF iff
\begin{itemize}
	\item $V \in \mathbb{C}^1$
	\item $V(0) = 0$ \\
	      $V(x) > 0 \quad \forall \quad x \in \mathbb{D} \backslash \{0\}$
	\item $\dot{V}(0) = 0$ \\
	      $\dot{V}(x) \leq 0 \quad \forall \quad x \in \mathbb{D} \backslash \{0\}$
\end{itemize}

\paragraph{Strict Lyapunov function (SLF)}
$V(x)$ is an SLF iff it is an LF and
\begin{itemize}
	\item $\dot{V}(x) < 0 \quad \forall \quad x \in \mathbb{D} \backslash \{0\}$
\end{itemize}

%%%%%%%%%%%%%%%%%%%%%%%%%%%%%%
\subsection{Autonomous systems}
%%%%%%%%%%%%%%%%%%%%%%%%%%%%%%
\paragraph{Theorem 4.1 (Direct Lyapunov method)}
\begin{itemize}
	\item If $\exists$ an LF for the origin, then the origin is stable.
	\item If $\exists$ an SLF for the origin, then the origin is asymptotically stable.
\end{itemize}

\paragraph{Theorem 4.2 (GAS)}
If $\exists$ an SLF $V$ for the origin and $V$ is radially unbounded, then the origin is globally asymptotically stable.

%%%%%%%%%%%%%%%%%%%%%%%%%%%%%%
\subsection{Invariance principle}
%%%%%%%%%%%%%%%%%%%%%%%%%%%%%%
\paragraph{Invariant sets} A set $M$ is an \emph{invariant set} w.r.t. $\dot{x} = f(x)$ if
\begin{equation}
	x(0) \in M \implies x(t) \in M \quad \forall t \in \mathbb{R}.
\end{equation}
(Any solution in $M$ stays in $M$ for all future and past.)

\paragraph{Theorem 4.4 (LaSalle's theorem)}
If $\exists \: V : \mathbb{D} \to \mathbb{R}$ such that
\begin{itemize}
	\item $V \in C^1$,
	\item $\exists \: c > 0$ such that $\Omega_c = \{x \in \mathbb{R}^n | V(x) \leq c \} \subset \mathbb{D}$ is bounded,
	\item $\dot{V}(x) \leq 0 \quad \forall \quad x \in \Omega_c$.
\end{itemize}
Let $E = \{ x \in \Omega_c | \dot{V}(x) = 0 \}$. Let $M$ be the largest invariant set in $E$. Then
\begin{equation}
	x(0) \in \Omega_c \implies x(t) \xrightarrow{t \to \infty} M.
\end{equation}

\paragraph{Corollary 4.1} Let $x^* = 0$ for $\dot{x} = f(x)$. If for an LF $V(x)$ we have $\dot{V}(x) \leq 0$ on $D$: Let $S = \{ x \in D | \dot{V}(x) = 0 \}$ and only $x(t) \equiv 0$ can stay in $S$, then $x = 0$ AS.

\paragraph{Corollary 4.2} If corollary 4.1 holds with $D = \mathbb{R}^n$, then $x = 0$ GAS.

%%%%%%%%%%%%%%%%%%%%%%%%%%%%%%
\subsection{Linear systems and linearisation}
%%%%%%%%%%%%%%%%%%%%%%%%%%%%%%
\paragraph{Theorem 4.7 (Lyapunov's indirect method)}
Let $x^* = 0$ for $\dot{x} = f(x)$ where $f : \mathbb{D} \rightarrow \mathbb{R}^n$ satisfies $f \in \mathbb{C}^1$ and $\mathbb{D}$ is a neighborhood of the origin. Let
\begin{equation}
	A = \pd{f}{x} (x) \evalat{x = 0}
\end{equation}
and $\lambda_i$ be the eigenvalues of $A$. Then
\begin{enumerate}
	\item $\Re \lambda_i < 0 \mbox{ for all } \lambda_i \implies x = 0$ AS.
	\item $\Re \lambda_i > 0 \mbox{ for any } \lambda_i \implies x = 0$ unstable.
\end{enumerate}

%%%%%%%%%%%%%%%%%%%%%%%%%%%%%%
\subsection{Comparison functions}
%%%%%%%%%%%%%%%%%%%%%%%%%%%%%%
\paragraph{Class $\mathcal{K}$ function}
A continuous function $\alpha : [0,a) \to [0, \infty)$ belongs to class $\mathcal{K}$ iff
\begin{itemize}
	\item it is stricly increasing,
	\item $\alpha(0) = 0$.
\end{itemize}

\paragraph{Class $\mathcal{K}_\infty$ function}
A continuous function $\alpha : [0,a) \to [0, \infty)$ belongs to class $\mathcal{K}_\infty$ iff
\begin{itemize}
	\item it is of class $\mathcal{K}$,
	\item $a = \infty$,
	\item $\alpha(r) \xrightarrow{r \to \infty} \infty$.
\end{itemize}

\paragraph{Class $\mathcal{KL}$ function}
A continunous function $\beta : [0,a) \times [0,\infty) \to [0,\infty)$ belongs to class $\mathcal{KL}$ if for each fixed $s$
\begin{itemize}
	\item $\beta(r,s)$ is a class $\mathcal{K}$ function w.r.t. $r$,
\end{itemize}
and for each fixed $r$
\begin{itemize}
	\item $\beta(r,s)$ is decreasing w.r.t. $s$,
	\item $\beta(r,s) \xrightarrow{s \to \infty} 0$.
\end{itemize}

%%%%%%%%%%%%%%%%%%%%%%%%%%%%%%
\subsection{Nonautonomous systems}
%%%%%%%%%%%%%%%%%%%%%%%%%%%%%%
\begin{equation}\label{eq:nonautonomous}
	\dot{x} = f(t,x), \quad f : [0,\infty) \times \mathbb{D} \to \mathbb{R}^n
\end{equation}

\paragraph{Decrescentness}
$V(t,x)$ is \emph{decrescent} iff
\begin{equation}
	\begin{rcases}
		V(t,0) =    0      \\
		V(t,x) \leq W_2(x)
	\end{rcases}
	\forall \: t \geq 0
	\mbox{, for some pos. def. }
	W_2(x).
\end{equation}

\paragraph{Theorem 4.8--4.9}
Let $V : [0,\infty) \times \mathbb{D} \to \mathbb{R}$ and $V \in \mathbb{C}^1$ for \eqref{eq:nonautonomous}. Then $x^* = 0$ is
\begin{center}
	\begin{tabular}{lllll}
		& S & US & UAS & GUAS \\
		\hline
		$V$ & PD & PD, decr. & PD, decr. & PD, decr., RU \\
		$\dot{V}$ & NSD & NSD & ND & ND \\
		$\forall x \in$ & $\mathbb{D}$ & $\mathbb{D}$ & $\mathbb{D}$ & $\mathbb{R}^n$
	\end{tabular}
\end{center}
(PD = positive definite, decr. = decrescent, RU = radially unbounded, NSD = negative semidefinite, ND = negative definite.)

\paragraph{Theorem 4.10 (exponential stability)}
If $\exists \: a, k_1, k_2, k_3 > 0$ such that
\begin{itemize}
	\item $V \in C^1$
	\item $k_1 \norm{x}^a \leq V(x) \leq k_2 \norm{x}^a \quad \forall \: x \in \mathbb{D}$
	\item $\dot{V}(x) \leq - k_3 \norm{x}^a \quad \forall \: x \in \mathbb{D}$
\end{itemize}
then $x = 0$ ES. If $\mathbb{D} = \mathbb{R}^n$, $x = 0$ is GES.

%%%%%%%%%%%%%%%%%%%%%%%%%%%%%%
\subsection{Converse theorems}
%%%%%%%%%%%%%%%%%%%%%%%%%%%%%%
\paragraph{Corollary 4.3}
The origin of $\dot{x} = f(x)$ is ES iff $A$ is Hurwitz, where
\begin{equation}
	A =
	\begin{bmatrix}
		\pd{f}{x}
	\end{bmatrix}
	\evalat{x=0}
	.
\end{equation}
