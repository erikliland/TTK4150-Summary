%!TEX root = TTK4150-Summary.tex
\section{Feedback linearisation}

%%%%%%%%%%%%%%%%%%%%%%%%%%%%%%
\subsection{Input-output linearisation}
%%%%%%%%%%%%%%%%%%%%%%%%%%%%%%
\begin{align}
	\dot{x} &= f(x) + g(x)u \label{eq:siso-x} \\
	y       &= h(x)         \label{eq:siso-y}
\end{align}

\paragraph{Definition 13.2 (relative degree)}
The relative degree $\rho$ of \eqref{eq:siso-x}--\eqref{eq:siso-y} is equal to the number of times $y$ must be differentiated until $u$ appears.

\paragraph{Theorem 13.1}
The system \eqref{eq:siso-x} can be input-output linearised if the relative degree is well defined in the region of interest $\mathbb{D}_0$.

\paragraph{Minimum phase}
The system \eqref{eq:siso-x}--\eqref{eq:siso-y} is \emph{minimum phase} if, for the zero-dynamics $\dot{\eta} = f_0(\eta,0)$, $\eta = 0$ is AS.

\subsubsection{Method}
The system is given by \eqref{eq:siso-x}--\eqref{eq:siso-y}. First determine relative degree $\rho$. Define
\begin{equation}
	\xi =
	\begin{bmatrix}
		\xi_1 \\ \vdots \\ \xi_\rho
	\end{bmatrix}
	=
	\begin{bmatrix}
		y \\ \vdots \\ y^{(\rho-1)}
	\end{bmatrix}
\end{equation}
so that
\begin{equation}
	\dot{\xi} =
	\begin{bmatrix}
		\dot{\xi}_1 \\ \vdots \\ \dot{\xi}_\rho
	\end{bmatrix}
	=
	\begin{bmatrix}
		\xi_2 \\ \vdots \\ L_f^\rho h + L_g L_f^{\rho-1} h \cdot u
	\end{bmatrix}
	.
\end{equation}
Then choose $n-\rho$ coordinates
\begin{equation}
	\eta =
	\begin{bmatrix}
		\eta_1 \\ \vdots \\ \eta_{n-\rho}
	\end{bmatrix}
\end{equation}
with the coordinate transformation
\begin{equation}
	z = T(x) =
	\begin{bmatrix}
		\eta_1 \\ \vdots \\ \eta_{n-\rho} \\ \xi_1 \\ \vdots \\ \xi_\rho
	\end{bmatrix}
\end{equation}
and choose $\eta$ such that
\begin{itemize}
	\item $T$ is a diffeomorphism,
	\item $L_g \eta_i = 0$,
	\item $\eta_i(0) = 0$.
\end{itemize}
We have then that
\begin{equation}
	\dot{\eta}_j = \pd{\eta_j}{x} \dot{x} = L_f \eta_j + \underbrace{L_g \eta_j}_{= 0} \cdot u = f_{0_j} (\eta_i, \xi_i)
\end{equation}
and we can write the system on normal form
\begin{equation}
	\begin{bmatrix}
		\dot{\eta}_1        \\
		\vdots              \\
		\dot{\eta}_{n-\rho} \\
		\dot{\xi}_1         \\
		\vdots              \\
		\dot{\xi}_\rho
	\end{bmatrix}
	=
	\begin{bmatrix}
		f_{0_1}(\eta_i, \xi_i)        \\
		\vdots                        \\
		f_{0_{n-\rho}}(\eta_i, \xi_i) \\
		\xi_2                         \\
		\vdots                        \\
		L_f^\rho h + L_g L_f^{\rho-1} h \cdot u
	\end{bmatrix}
	.
\end{equation}
Then choose $u$ to cancel the nonlinearities:
\begin{equation}
	u = \frac{1}{L_g L_f^{\rho-1} h} (-L_f^\rho h + v)
\end{equation}
This leads to
\begin{gather}
	\dot{\eta} = f_0(\eta,\xi) \\
	\dot{\xi} = 
	\begin{bmatrix}
		\xi_2 \\ \vdots \\ v
	\end{bmatrix}
\end{gather}
Then analyse the zero-dynamics---the internal dynamics when the output is kept zero by the input:
\begin{equation}
	\dot{\eta} = f_0(\eta,0)
\end{equation}