%!TEX root = TTK4150-Summary.tex
\section{Stability of perturbed systems}
We consider perturbed systems on the form
\begin{equation}
	\dot{x} = f(t,x) + g(t,x)
\end{equation}
with \emph{nominal} systems
\begin{equation}
	\dot{x} = f(t,x)
	.
\end{equation}

%%%%%%%%%%%%%%%%%%%%%%%%%%%%%%
\subsection{Vanishing perturbation}
%%%%%%%%%%%%%%%%%%%%%%%%%%%%%%
\paragraph{Lemma 9.1}
\begin{itemize}
	\item The origin is an ES equilibrium of the nominal system.
	\item $V(t,x)$ is an LF of the nominal system, and satisfies
		\begin{equation}
			c_1 \norm{x}^2 \leq V(t,x) \leq c_2 \norm{x}^2
		\end{equation}
		and
		\begin{equation}
			\norm{\pd{V}{x}} \leq c_4 \norm{x}
			.
		\end{equation}
	\item The perturbation $g(t,x)$ satisfies
		\begin{equation}
			\norm{g(t,x)} \leq \gamma \norm{x}, \quad \gamma < \frac{c_3}{c_4}
			.
		\end{equation}
\end{itemize}
Then $x^* = 0$ of the perturbed system is ES. If the assumptions hold globally, $x = 0$ is GES.


%%%%%%%%%%%%%%%%%%%%%%%%%%%%%%
\subsection{Nonvanishing perturbation}
%%%%%%%%%%%%%%%%%%%%%%%%%%%%%%