%!TEX root = TTK4150-Summary.tex
\section{Just math}

%%%%%%%%%%%%%%%%%%%%%%%%%%%%%%
\subsection{Metric norms}
%%%%%%%%%%%%%%%%%%%%%%%%%%%%%%
\paragraph{General $p$-norm}
\begin{equation}
	||x||_p
	=
	\left(
		\sum_{i=1}^{n} |x_i|^p
	\right)
	^{1/p}
\end{equation}

\paragraph{Taxicab norm ($p=1$)}
\begin{equation}
	||x||_1
	=
	\sum_{i=1}^{n} |x_i|
\end{equation}

\paragraph{Euclidean norm ($p=2$)}
\begin{equation}
	||x||_2
	=
	\sqrt{x_1^2 + \cdots + x_n^2}
\end{equation}

%%%%%%%%%%%%%%%%%%%%%%%%%%%%%%
\subsection[\texorpdfstring{Boundedness and $\mathcal{L}_p$-norms}
	{Boundedness and Lp-norms}]
	{Boundedness and $\mathcal{L}_p$-norms}
%%%%%%%%%%%%%%%%%%%%%%%%%%%%%%
\paragraph{$\mathcal{L}_p$-norm}
\begin{equation}
	||f||_p
	=
	\left(
		\int_a^b |f(\tau)|^p \dif \tau
	\right)^{1/p}
\end{equation}

\paragraph{$\mathcal{L}_\infty$-norm}
\begin{equation}
	||f||_\infty
	=
	\sup_{a \leq t \leq b} |f(t)|
\end{equation}

\paragraph{Boundedness}
\begin{equation}
	f \in \mathcal{L}_p \Leftrightarrow \norm{f}_p < \infty
\end{equation}

%%%%%%%%%%%%%%%%%%%%%%%%%%%%%%
\subsection{Properties of norms}
%%%%%%%%%%%%%%%%%%%%%%%%%%%%%%
\paragraph{Hölder's inequality}
\begin{equation}
	\norm{fg}_1 \leq \norm{f}_p \norm{g}_q
\end{equation}
with $\frac{1}{p} + \frac{1}{q} = 1$.

%%%%%%%%%%%%%%%%%%%%%%%%%%%%%%
\subsection{Matrix properties}
%%%%%%%%%%%%%%%%%%%%%%%%%%%%%%
\paragraph{Singular} A matrix is \emph{singular} iff its determinant is zero.

\paragraph{Skew-symmetry} A matrix $A$ is \emph{skew-symmetric} iff
\begin{equation}
	-A = A\T
	.
\end{equation}

\paragraph{Jacobian} The \emph{Jacobian} matrix is defined by
\begin{equation}
	J
	=
	\od{f}{x}
	=
	\begin{bmatrix}
		\pd{f_1}{x_1} & \cdots & \pd{f_1}{x_n} \\
		\vdots        & \ddots & \vdots        \\
		\pd{f_m}{x_1} & \cdots & \pd{f_m}{x_n}
	\end{bmatrix}
	.
\end{equation}

\paragraph{Hurwitz}
A matrix $A$ is \emph{Hurwitz} if all eigenvalues of $A$ satisfy $\Re \lambda_i < 0$.

\paragraph{Positive definite} A matrix being \emph{positive definite} is equivalent to
\begin{itemize}
	\item all its eigenvalues being positive,
	\item all its leading principal minors being positive.
\end{itemize}
In addition, we have
\begin{equation}
	\lambda\sub{min}(H) x\T x \leq x\T H x \leq \lambda\sub{max}(H) x\T x
\end{equation}
for a positive definite $x\T H x$.