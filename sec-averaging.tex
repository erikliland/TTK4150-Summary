%!TEX root = TTK4150-Summary.tex
\section{Perturbation theory and averaging}
\begin{equation}\label{eq:perturbed}
	\dot{x} = f(x) + \epsilon g(t,x,\epsilon)
\end{equation}

%%%%%%%%%%%%%%%%%%%%%%%%%%%%%%
\subsection{Periodic perturbation of autonomous systems}
%%%%%%%%%%%%%%%%%%%%%%%%%%%%%%
\paragraph{Definition of $P_\epsilon(x)$:}
$\phi(t;t_0,x_0,\epsilon)$ is the solution of \eqref{eq:perturbed} that starts at $(t_0,x_0)$. $P_\epsilon(x)$ is
\begin{equation}
	P_\epsilon(x) = \phi(T;0,x,\epsilon)
\end{equation}

\paragraph{Lemma 10.1}
The system \eqref{eq:perturbed} has a T-periodic solution iff
\begin{equation}
	x = P_\epsilon(x)
\end{equation}
has a solution.

%%%%%%%%%%%%%%%%%%%%%%%%%%%%%%
\subsection{Averaging}
%%%%%%%%%%%%%%%%%%%%%%%%%%%%%%
